% !TEX root = ../main.tex

\hitszsetup{
  %******************************
  % 注意:
  %   1. 配置里面不要出现空行
  %   2. 不需要的配置信息可以删除
  %******************************
  %
  %=====
  % 秘级
  %=====
  statesecrets={公开},
  natclassifiedindex={TM301.2},
  intclassifiedindex={62-5},
  %
  %=========
  % 中文信息
  %=========
  ctitleone={带时间敏感性的无人机网络},%本科生封面使用
  ctitletwo={扫描覆盖算法设计与实现},%本科生封面使用
  ctitlecover={带时间敏感性的无人机网络\\扫描覆盖算法设计与实现},%放在封面中使用,自由断行
  ctitle={带时间敏感性的无人机网络扫描覆盖算法设计与实现},%放在原创性声明中使用
  % csubtitle={一条副标题}, %一般情况没有,可以注释掉
  cxueke={工学},
  cpostgraduatetype={学术},
  csubject={计算机科学与技术},
  % csubject={机械工程},
  caffil={计算机科学与技术学院},
  % caffil={哈尔滨工业大学(深圳)},
  cauthor={胡聪},
  csupervisor={堵宏伟\enspace 副教授},
  % cassosupervisor={某某某 教授}, % 副指导老师
  % ccosupervisor={某某某 教授}, % 联合指导老师
  % 日期自动使用当前时间,若需指定按如下方式修改:
  cdate={2022年6月},
  % 指定第二页封面的日期,即答辩日期
  cdatesecond={2022年06月09日},
  cstudentid={180110505},
  % cstudenttype={同等学力人员}, %非全日制教育申请学位者
  %(同等学力人员)、(工程硕士)、(工商管理硕士)、
  %(高级管理人员工商管理硕士)、(公共管理硕士)、(中职教师)、(高校教师)等
  %
  %
  %=========
  % 英文信息
  %=========
  etitle={Research on robot intelligent grasping based on Neural Network},
  esubtitle={This is the sub title},
  exueke={Engineering},
  esubject={Mechanical Engineering},
  eaffil={Harbin Institute of Technology, Shenzhen},
  eauthor={Jingxuan Yang},
  esupervisor={Prof. XXX},
  % eassosupervisor={XXX},
  % 日期自动生成,若需指定按如下方式修改:
  edate={June, 2020},
  estudenttype={Master of Engineering},
  %
  % 关键词用“英文逗号”分割
  ckeywords={无线传感器网络, 无人机, 扫描覆盖, 贪心算法, 遗传算法},
  ekeywords={WSN, UAV, sweep coverage, greedy algorithm, genetic algorithm},
}

% 中文摘要
\begin{cabstract}

  近年来,随着无人机技术的高速发展,其在航拍、物流、农业、公共安全和救援等领域有着非常广阔的发展前景,在这些领域中,常常需要多架无人机对特定区域进行扫描覆盖任务。
  由于无人机的路线规划方案对无人机的扫描覆盖率有着较大的影响,因此针对无人机的路径优化研究就显得非常必要和有意义。


  本文根据国内外研究现状,以灾害发生后的应急救灾场景作为背景,着重考虑了无人机的特点和救援任务对时间的要求等因素,
  构建了多目标带时间窗的无人机扫描覆盖路径问题模型,以实现扫描覆盖率高和成本较低等多个目标。
  接下来分别采用贪心算法和遗传算法对该模型进行求解,最后通过结果分析验证了模型和算法的有效性。


\end{cabstract}

% 英文摘要
\begin{eabstract}

  In recent years, with the rapid development of unmanned aerial vehicle technology,
   it has a very broad development prospects in the areas of aerial photography, logistics, 
   agriculture, public safety and rescue, in which multiple unmanned aerial vehicles are often required to scan and cover specific areas. 
   Because the UAV's route planning scheme has a large impact on the UAV's scan coverage, it is necessary and meaningful to study the UAV's route optimization.

  
  In this paper, based on the research status at home and abroad, taking the emergency relief scenario after a disaster as the background, 
  and considering the characteristics of unmanned aerial vehicles and the time requirements of rescue tasks, 
  a multi-objective model of unmanned aerial scanning coverage path problem with time windows is constructed to achieve multiple goals such as high scan coverage and low cost. 
  Next, the greedy algorithm and genetic algorithm are used to solve the model. Finally, the validity of the model and the algorithm is verified by the result analysis.

\end{eabstract}
