% !TEX root = ../main.tex

\hitszsetup{
  %******************************
  % 注意:
  %   1. 配置里面不要出现空行
  %   2. 不需要的配置信息可以删除
  %******************************
  %
  %=====
  % 秘级
  %=====
  statesecrets={公开},
  natclassifiedindex={TM301.2},
  intclassifiedindex={62-5},
  %
  %=========
  % 中文信息
  %=========
  ctitleone={带时间敏感性的无人机网络},%本科生封面使用
  ctitletwo={扫描覆盖算法设计与实现},%本科生封面使用
  ctitlecover={带时间敏感性的无人机网络\\扫描覆盖算法设计与实现},%放在封面中使用,自由断行
  ctitle={带时间敏感性的无人机网络扫描覆盖算法设计与实现},%放在原创性声明中使用
  % csubtitle={一条副标题}, %一般情况没有,可以注释掉
  cxueke={工学},
  cpostgraduatetype={学术},
  csubject={计算机科学与技术},
  % csubject={机械工程},
  caffil={计算机科学与技术学院},
  % caffil={哈尔滨工业大学(深圳)},
  cauthor={胡聪},
  csupervisor={堵宏伟~副教授},
  % cassosupervisor={某某某 教授}, % 副指导老师
  % ccosupervisor={某某某 教授}, % 联合指导老师
  % 日期自动使用当前时间,若需指定按如下方式修改:
  cdate={2022年6月},
  % 指定第二页封面的日期,即答辩日期
  cdatesecond={2022年06月XX日},
  cstudentid={180110505},
  % cstudenttype={同等学力人员}, %非全日制教育申请学位者
  %(同等学力人员)、(工程硕士)、(工商管理硕士)、
  %(高级管理人员工商管理硕士)、(公共管理硕士)、(中职教师)、(高校教师)等
  %
  %
  %=========
  % 英文信息
  %=========
  etitle={Research on robot intelligent grasping based on Neural Network},
  esubtitle={This is the sub title},
  exueke={Engineering},
  esubject={Mechanical Engineering},
  eaffil={Harbin Institute of Technology, Shenzhen},
  eauthor={Jingxuan Yang},
  esupervisor={Prof. XXX},
  % eassosupervisor={XXX},
  % 日期自动生成,若需指定按如下方式修改:
  edate={June, 2020},
  estudenttype={Master of Engineering},
  %
  % 关键词用“英文逗号”分割
  ckeywords={无人机, 扫描覆盖, 贪心算法, 遗传算法},
  ekeywords={UAV, sweep coverage, greedy algorithm, genetic algorithm},
}

% 中文摘要
\begin{cabstract}
近年来,随着无人机技术的高速发展,其在航拍、物流、农业、公共安全和救援等领域有着非常广阔的发展前景,在这些领域中,常常需要多架无人机对特定区域进行扫描覆盖任务。由于无人机的路线规划方案对无人机的扫描覆盖率有着较大的影响,因此针对无人机的路径优化研究就显得非常必要和有意义。


本文根据国内外研究现状,以灾害发生后的应急救灾场景作为背景,着重考虑了无人机的特点和救援任务对时间的要求等因素,构建了多目标带时间窗的无人机扫描覆盖路径问题模型,以实现较高有效覆盖率的目标。接下来针对贪心算法,修改了其成本函数的构造方式,改进了无人机的路径规划流程,设计了GCSAWL算法;同时针对遗传算法,在流程中增加了多轮筛选、优秀个体自交等过程,使其流程更符合自然进化的规律,设计了GAWS算法。实验结果表明,改进后的GCSAWL算法相较GCS算法在有效覆盖率方面平均提升约5\%至6\%,运输成本与时间方面平均降低约3\%至4\%,而GAWS算法相较GCS算法在有效覆盖率方面平均提升约8\%至9\%,运输成本与时间方面平均降低约5\%至7\%。最后,还设计了一套无人机紧急救援模拟系统,实现了无人机路径规划与调度情况的直观显示。
\end{cabstract}

% 英文摘要
\begin{eabstract}
In recent years, with the rapid development of UAV technology, it has a very broad development prospect in the fields of aerial photography, logistics, agriculture, public safety and rescue. Scan overwrite tasks. Since the route planning scheme of the UAV has a great influence on the scanning coverage of the UAV, it is very necessary and meaningful to study the path optimization of the UAV.


According to the current research situation at home and abroad, taking the emergency rescue scene after the disaster as the background, focusing on factors such as the characteristics of the UAV and the time requirements of the rescue mission, a multi-target UAV scanning coverage path with a time window is constructed. problem model to achieve the goal of high effective coverage. Next, for the greedy algorithm, the construction method of its cost function was modified, the path planning process of the UAV was improved, and the GCSAWL algorithm was designed; at the same time, for the genetic algorithm, multiple rounds of screening and excellent individual self-crossing were added to the process , so that the process is more in line with the law of natural evolution, and the GAWS algorithm is designed. The experimental results show that the improved GCSAWL algorithm has an average increase of about 5\% to 6\% in effective coverage compared with the GCS algorithm, and an average reduction of about 3\% to 4\% in transportation cost and time, while the GAWS algorithm is more effective than the GCS algorithm. In terms of coverage, the average increase is about 8\% to 9\%, and the average transportation cost and time are reduced by about 6\% to 7\%. Finally, a set of UAV emergency rescue simulation system is also designed, which realizes the intuitive display of UAV path planning and scheduling.
\end{eabstract}
