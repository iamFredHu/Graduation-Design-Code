% !TEX root = ../main.tex

% 中英标题:\chapter{中文标题}[英文标题]
\chapter{基于改进的贪心算法的模型求解方案}[Model Solving Scheme Based on Improved Greedy Algorithm]

\section{算法的基本思想}[Algorithm th]
使用贪心算法对模型进行求解的基本思想是,在一次救援任务中依次为每架无人机生成扫描覆盖路径,每条扫描覆盖路径的起点和终点均为基地$B$。在路径规划过程中,我们设计了一个成本函数来计算访问每个兴趣点的成本。
该成本函数考虑了访问兴趣点所需的时间、该兴趣点的时间敏感性$T_s$、当前无人机的扫描覆盖进度和载重等因素。在路径规划时,我们采用贪婪策略,每次都选择覆盖成本最低的兴趣点,从而为当前无人机获得最佳的扫描路径。

\section{算法的细节}[Details of the algorithm]
在无人机出发之前,首先需要检查兴趣点的时间敏感性$T_s$是否合理。如果兴趣点$i$的时间敏感性$ts_i$比无人机从基地$B$至该兴趣点的直线飞行时间$\frac{d_{iB}}{v}$小,则说明无论如何无人机都无法在准确时间内对该兴趣点进行访问,
需要重新检查时间敏感性。接下来,逐个对$m$架无人机规划扫描路径$O=\lbrace o_1, o_2, \cdots ,o_m \rbrace$进行规划。第$k$架无人机的初始扫描路径$O_k$为$\emptyset$。变量$T_k$用于记录无人机在起飞后所花费的时间,其初始值
为无人机在上升阶段所花费的时间$T_a$。对于$P$中的每个兴趣点$p_i$,我们首先判断这个兴趣点是否满足兴趣点可访问的三个条件。第一个条件是当无人机飞到兴趣点$p_i$时,该兴趣点仍然存活,也就是说时间在可接受范围$(1+e) \cdot ts_i$内。
第二个条件是,当无人机执行完任务后,剩余的续航里程可以让无人机返回至基地$B$进行充电和维护等作业。第三个条件是无人机在执行任务时不能超重运输。只有当这三个条件同时满足时,我们才说兴趣点$p_i$是可访问的。


然后,我们设计了一个成本函数来评估无人机访问$p_i$的成本:
$$ \varphi = \frac{T_k}{T_{max}}$$
$$ C_i = \alpha t_i + \beta (ts_i - T_k)^\varphi + \gamma m_i $$


系数$\alpha$反映当前无人机的扫描覆盖任务完成进度。$(ts_i - T_k)$表示兴趣点$p_i$的剩余访问时间,扫描覆盖任务越接近结束,访问兴趣点的优先级越高,剩余的访问时间越短。$m_i$代表的是兴趣点$p_i$所需求的救援物资质量。
$\alpha,\beta,\gamma$是系数,负责调节三个影响因素的权重。在成本函数中,距离当前位置近、剩余可访问时间短且载重要求低的兴趣点访问成本最低,将首先被访问。如果该兴趣点不可访问,则会将$C_i$设置为$+\infty$。计算完所有兴趣点
的成本后,选择访问成本最小的兴趣点$p_j$,然后将$p_j$添加到路径$O_k$中,并将其从$P$中删除,更新计时器$T_k$,并判断兴趣点$p_j$的准时覆盖情况和有效覆盖情况。


如果$p_j$不存在,即当前$P$中所有的兴趣点成本均为$+\infty$,则意味着没有兴趣点可以被继续覆盖,在这种情况下,我们还可以要求无人机在不超重且续航里程足够的情况下,访问较近的已超时但未被访问的兴趣点,实现对所有兴趣点的尽可能的覆盖。
完成任务后,无人机$u_k$应返回基地$B$,当$u_k$返回基地或已经没有剩余的未被访问的兴趣点时,$u_k$的路径规划完成。


最后,在全部$m$架无人机的路径都被规划完成后,根据第二章中提到的公式计算有效覆盖率$R_e$和准时覆盖率$R_o$,算法结束。

\section{算法的伪代码}[Pseudocode of the algorithm]
\begin{algorithm}[H]
\caption{改进的贪心算法}
\LinesNumbered %要求显示行号
\KwIn{\rm video \ \it X,\ \rm hole \ \it H,\ \rm validity \ \it V \\}
\KwOut{\rm complection \ video \it Y \\ }
\For{condition}{
	only if\;
	\If{condition}{
		1\;
	}
}
\end{algorithm}

\section{算法的复杂度分析}[Complexity Analysis of Algorithm]
在本算法中,计算$P$中所有兴趣点的访问成本的部分复时间复杂度为$O(n)$,因为$P$有$n$个兴趣点。同样,将兴趣点添加到无人机的扫描路径的过程,其时间复杂度也为$O(n)$。最后,在规划$m$架无人就的扫描路径时,迭代次数为$m$,故
本算法的时间复杂度为$O(mn^2)$。

\section{算法的改进}[Improved algorithms]

\section{本章小结}[Brief summary]
在本章中,我们主要介绍了改进的贪心算法的基本思想和算法细节,同时对该算法的复杂性进行了分析。