% !TEX root = ../main.tex

\chapter{仿真实验及结果分析}[Simulation Experiment and Result Analysis]

\section{参数设置}[Parameter setting]

本次实验的仿真的平台信息如下:

\begin{table}[h]
\begin{center}
\caption{仿真实验所用平台信息} 
\begin{tabular}{c||c}
\hline
平台 & 个人PC \\
\hline
操作系统 & Microsoft Windows 11 x64 专业版 21H2 \\
\hline
语言 & Python 3.9.12 \\
\hline
CPU型号 & AMD Ryzen 7 5800H \\
\hline
内存 & 16.00 GB \\
\hline
IDE & PyCharm 2021.3.1  \\ 
\hline
\end{tabular}
\end{center}
\end{table}

\subsection{实验参数}[Parameter setting]
在实验的模拟中,模拟的目标区域是边长为$50km$的正方形,无人机基地$B$位于该区域的中心。在该区域内兴趣点随机分布,每个兴趣点的需求物资量也随机产生。无人机的最大续航时间为$180$分钟,最大载重为$6kg$,其中从基地起飞和降落的时间忽略不计。
每架无人机的移动速度$v$设置为$25m/s$。为了研究不同变量,即无人机的数量$m$、兴趣点的数量$n$和时间敏感性范围$T_s$对时延的影响,我们设计了以下三个场景的仿真实验:
\begin{itemize}
	\item [(1)] $n = 100$,$T_s$的随机范围为$50$分钟至$140$分钟,$m$的范围为$0$至$10$;
	\item [(2)] $m = 5$,$T_s$的随机范围为$50$分钟至$140$分钟,,$n$的范围为$0$至$400$;
 	\item [(3)] $m = 5$,$n = 100$,$T_s$的随机范围为$30$分钟至$100$分钟,$50$分钟至$140$分钟和
$70$分钟至$160$分钟三组。
\end{itemize}
\subsection{算法参数}[Parameter setting]
在使用改进的贪心算法进行实验仿真时,没有需要设置的算法参数。


在使用改进的遗传算法进行实验仿真时,交叉概率设置为$0.9$,而变异概率设置为$0.1$。
\section{性能对比与结果分析}[Algorithm Comparison]
\begin{figure}[h]
	\centering
	\includegraphics[width = 0.5\textwidth]{fg0_bg}
	\caption{$n = 100$,$T_s$的随机范围为$50$分钟至$140$分钟}
	\label{fg501}
\end{figure}
场景1:\figref{fg501}显示了有效覆盖率$R_e$和准时覆盖率$R_0$随无人机$m$的数量不断变化的过程。实线
代表的是有效覆盖率$R_e$,虚线代表的是准时覆盖率$R_0$。实验结果表明,不断改变无人机的数量,改进后的算法的$R_e$和$R_0$均高于改进前。
在无人机数量较少时,改进后的贪心算法其$R_e$和$R_0$与改进后的遗传算法接近,但是当无人机数量增加时,改进后的遗传算法的优势变得较为明显。
\begin{figure}[h]
\centering
\includegraphics[width = 0.5\textwidth]{fg0_bg}
\caption{$m = 5$,$T_s$的随机范围为$50$分钟至$140$分钟}
\label{fg502}
\end{figure}


场景2:\figref{fg502}显示了有效覆盖率$R_e$和准时覆盖率$R_0$随兴趣点$n$的数量不断变化的过程。实线
代表的是有效覆盖率$R_e$,虚线代表的是准时覆盖率$R_0$。实验结果表明,不断改变兴趣点的数量,改进后的算法的$R_e$和$R_0$均高于改进前。
且改进后的遗传算法其$R_e$和$R_0$始终高于改进后的贪心算法。我们也要注意到,随着兴趣点数量的不断增加,四种算法的$R_e$和$R_0$均在下降,这是因为无人机的数量是固定的,其可以覆盖的兴趣点的数量也是有限的。
实验结果显示,在这四种算法中,改进后的遗传算法性能最为优秀。


场景3:\figref{fg503}显示了有效覆盖率$R_e$与时间敏感性$T_s$范围之间的关系。当$T_s$在30分钟至120分钟时,时间敏感性的最小值较小,这对于无人机的覆盖要求较高,因此有效覆盖率$R_e$也会相应更低。然而在这三张图
中,改进后的遗传算法的$R_e$始终优于其余三种算法。实验结果表明,在更恶劣的条件下,改进后的遗传算法优势更加明显。
\begin{figure}[h]
	\centering
	\includegraphics[width = 0.5\textwidth]{fg0_bg}
	\caption{$m = 5$,$n = 100$,$T_s$的随机范围分为三种情况(占位图,实际上有3张图)}
	\label{fg503}
	\end{figure}

\section{本章小结}[Brief summary]
本章主要对带时间敏感性的无人机网络扫描覆盖问题进行了实验测试。根据对现实生活中实际情况的调查和研究,
设置了实验参数,并用两种算法进行了求解。通过对实验结果的分析,我们发现两种改进后的算法均能取得较好的结果,
且改进后的遗传算法相较改进后的贪心算法有着更好的优化效果。

