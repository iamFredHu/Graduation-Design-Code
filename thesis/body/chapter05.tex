% !TEX root = ../main.tex

\chapter{仿真实验及结果分析}[Simulation Experiment and Result Analysis]

\section{仿真实验的参数设置}[Parameter setting]

本次实验的仿真于个人PC上完成,操作系统为Microsoft Windows 11 x64 专业版 21H2,内存大小为16.00GB,CPU型号为AMD Ryzen 7 5800H,使用编程语言版本为Python 3.9.12,IDE版本为PyCharm 2021.3.1。


在实验的模拟中,模拟的目标区域是边长为50km的正方形,无人机基地$B$位于该区域的中心。在该区域内兴趣点随机分布,每个兴趣点的需求物资量也随机产生。无人机的最大续航时间为180分钟,最大载重为100kg,
其中从基地起飞和降落的时间忽略不计。每架无人机的移动速度$v$设置为25m/s。为了更好地得知算法的性能,本文设置了实验组(一):
\begin{itemize}
	\item [(1)] 比较在无人机数量$m$改变,兴趣点数量$n$不变,兴趣点的时间敏感性$T_s$随机范围固定,兴趣点的物资需求$S$随机范围固定的条件下,带载重的贪婪成本选择算法、经典遗传算法和带自交的遗传算法共三种算法的有效覆盖率;
    \item [(2)] 比较在兴趣点数量$n$改变,无人机数量$m$不变,兴趣点的时间敏感性$T_s$随机范围固定,兴趣点的物资需求$S$随机范围固定的条件下,带载重的贪婪成本选择算法、经典遗传算法和带自交的遗传算法共三种算法的有效覆盖率;
    \item [(3)] 比较在兴趣点的时间敏感性$T_s$随机范围改变,兴趣点数量$n$不变,无人机数量$m$不变,兴趣点的物资需求$S$随机范围固定的条件下,带载重的贪婪成本选择算法、经典遗传算法和带自交的遗传算法共三种算法的有效覆盖率;
\end{itemize}


为了更好地比较三种算法间的性能差异,还设置了实验组(二):
\begin{itemize}
	\item [(1)] 在其他参数不变的情况下,比较带载重的贪婪成本选择算法和改进前后遗传算法结果的无人机总飞行里程;
    \item [(2)] 在其他参数不变的情况下,比较带载重的贪婪成本选择算法和改进前后遗传算法结果的无人机总飞行时间。
\end{itemize}


为了更好地比较经典遗传算法和带自交的遗传算法间的性能差异,还设置了实验组(三):
\begin{itemize}
	\item [(1)] 在其他参数不变的情况下,逐次增加遗传算法的迭代次数,比较改进前后遗传算法的准时覆盖率和有效覆盖率;
 	\item [(2)] 在其他参数不变的情况下,逐次增加遗传算法的迭代次数,比较改进前后遗传算法结果中违反载重约束的路径数量变化。
\end{itemize}
\section{GCSAWL算法与GAWS算法的性能对比与实验结果分析}[Algorithm Comparison]
实验组(一):待救援区域为边长50km的正方形,基地$B$位于$(0,0)$,无人机的飞行速度为25m/s,最大载重为100kg,遗传算法的迭代次数为50次。
\begin{itemize}
    \item [(1)] 兴趣点的数量$n$设置为100,兴趣点的时间敏感性$T_s$随机范围为(60mins,150mins),物资需求$S$随机范围为(5kg,15kg),无人机的数量$m$设置为0至10,实验结果如\figref{fg501}所示:
    \begin{figure}[H]
        \begin{center}
            \begin{tikzpicture}
                \begin{axis}
                    %坐标系配置,只显示左边坐标轴
                    [axis lines = left, xlabel = {无人机数量(架)}, ylabel = {有效覆盖率},tick align=outside,legend style={at={(1.5,0.75)},anchor=north}]
                    %绘制曲线-2
                    \addplot+[smooth,green,mark=square]
                    coordinates{
                        (0,0) (1,0.1600) (2,0.2200)
                        (3,0.3000) (4,0.3700)(5,0.4700)
                        (6,0.5600) (7,0.6500)(8,0.7500)
                        (9,0.84) (10,0.9200)
                    };
                    \addlegendentry{带载重的贪婪成本选择算法}; %绘制标注
                    %绘制曲线-3
                    \addplot+[smooth,blue,mark=x]
                    coordinates{
                        (0,0) (1,0.1050) (2,0.1800)
                        (3,0.2932) (4,0.3945)(5,0.4987)
                        (6,0.6021) (7,0.7105)(8,0.7834)
                        (9,0.87) (10,0.96)
                    };
                    \addlegendentry{经典遗传算法}; %绘制标注
                    %绘制曲线-4
                    \addplot+[smooth,cyan,mark=+]
                    coordinates{
                        (0,0) (1,0.1100) (2,0.1900)
                        (3,0.3128) (4,0.4373)(5,0.5191)
                        (6,0.6323) (7,0.7491)(8,0.8252)
                        (9,0.9200) (10,1)
                    };
                    \addlegendentry{带自交的遗传算法}; %绘制标注
                \end{axis}
            \end{tikzpicture}
        \end{center}
        \caption{三种算法的覆盖率随无人机数量的变化情况}
        \label{fg501}
    \end{figure}
    实验结果说明,在无人机数量较低时,带载重的贪婪成本选择算法的有效覆盖率$R_e$结果比其余两种遗传算法要高,但当无人机数量增加后,带自交的遗传算法的有效覆盖率$R_e$平均较带载重的贪婪成本选择算法提高约7\%至9\%,较经典遗传算法平均提高约5\%。
    \item [(2)]无人机的数量$m$设置为10,兴趣点的时间敏感性$T_s$随机范围为(60mins,150mins),
    物资需求$S$随机范围为(5kg,15kg),兴趣点的数量$n$设置为25、50、75、100、150、200、250、300、350、400、450和500,实验结果如\figref{fg502}所示:
    \begin{figure}[H]
        \begin{center}
            \begin{tikzpicture}
                \begin{axis}
                    %坐标系配置,只显示左边坐标轴
                    [xlabel = {兴趣点数量(个)}, ylabel = {有效覆盖率},tick align=outside,legend style={at={(1.5,0.75)},anchor=north},ytick={0,0.1,0.2,0.3,0.4,0.5,0.6,0.7,0.8,0.9,1}]
                    %绘制曲线-2
                    \addplot+[smooth,green,mark=square]
                    coordinates{
                        (0,1) (25,1) (50,1)
                        (75,1) (100,0.9500)(150,0.6400)
                        (200,0.495) (250,0.408)(300,0.34)
                        (350,0.3029) (400,0.2450)(450,0.2267)(500,0.1960)
                    };
                    \addlegendentry{带载重的贪婪成本选择算法}; %绘制标注
                    %绘制曲线-3
                    \addplot+[smooth,blue,mark=x]
                    coordinates{
                        (0,1) (25,1) (50,1)
                        (75,1) (100,0.9800)(150,0.6867)
                        (200,0.5050) (250,0.430)(300,0.3633)
                        (350,0.3000) (400,0.2500)(450,0.2500)(500,0.2200)
                    };
                    \addlegendentry{经典遗传算法}; %绘制标注
                    %绘制曲线-4
                    \addplot+[smooth,cyan,mark=+]
                    coordinates{
                        (0,1) (25,1) (50,1)
                        (75,1) (100,1)(150,0.7178)
                        (200,0.5508) (250,0.4609)(300,0.3726)
                        (350,0.3258) (400,0.2784)(450,0.2763)(500,0.2406)
                    };
                    \addlegendentry{带自交的遗传算法}; %绘制标注
                \end{axis}
            \end{tikzpicture}
        \end{center}
        \caption{三种算法的覆盖率随兴趣点数量的变化情况}
        \label{fg502}
    \end{figure}
    从实验结果可以看出,随着兴趣点数量的增加,少量的无人机难以对大量的兴趣点实现有效覆盖,因此有效覆盖率$R_e$的整体趋势是下降的。但是,尽管三种算法的有效覆盖率$R_e$
    都在不断下降,带自交遗传算法的有效覆盖率$R_e$始终高于其余两种算法,体现出其性能的优越性。
    \item [(3)]无人机的数量$m$设置为5,兴趣点的数量$n$设置为100,兴趣点的时间敏感性$T_s$随机范围为(20mins,120mins),(60mins,150mins)和(90mins,180mins)三组,物资需求$S$随机范围为(5kg,15kg),实验结果如表\ref{table2}所示:
    \begin{table}[htbp]
        \vspace{0.5em}\centering\wuhao
        \caption{三种算法的有效覆盖率对比}\label{table2}
        \begin{tabular}{cccc}
        \toprule[1.5pt]
        算法名称 & (20mins,120mins) & (60mins,150mins) & (90mins,180mins) \\
        \midrule[1.5pt]
        带载重的贪婪成本算法 & 54.21\% & 62.18\% & 69.54\%\\
        经典遗传算法 & 56.78\% & 63.47\% & 71.98\% \\
        带自交的遗传算法 & 59.23\% & 67.32\% & 73.45\% \\
        \bottomrule[1.5pt]
        \end{tabular}
        \end{table}
    
    
        实验结果说明了有效覆盖率$R_e$与时间敏感性范围$T_s$之间的关系。时间敏感性范围(20mins,120mins)说明有一些兴趣点的最低时间可能只有20分钟,这意味着无人机很难去覆盖这些兴趣点。
    而从结果来看,带自交的遗传算法在(20mins,120mins)、(60mins,150mins)和(90mins,180mins)三个区间内有效覆盖率$R_e$均比带载重的贪婪成本算法和经典遗传算法高。
\end{itemize}


实验组(二):待救援区域为边长50km的正方形,基地$B$位于$(0,0)$,无人机的数量为5架,兴趣点的数量为50,无人机的飞行速度为25m/s,最大载重为100kg,兴趣点的时间敏感性随机范围为(50mins,140mins),物资需求随机范围为(5kg,15kg)。遗传算法的迭代次数为50次。


实验结果如表\ref{table3}所示:
\begin{table}[htbp]
    \vspace{0.5em}\centering\wuhao
    \caption{三种算法的无人机总飞行里程和时间对比}\label{table3}
    \begin{tabular}{ccc}
    \toprule[1.5pt]
    算法名称 & 总飞行时间(mins) & 总飞行里程(km) \\
    \midrule[1.5pt]
    带载重的贪婪成本选择算法 & 266.88 & 400.33 \\
    经典遗传算法 & 243.05 & 364.58 \\
    带自交的遗传算法 & 234.63 & 351.95 \\
    \bottomrule[1.5pt]
    \end{tabular}
    \end{table}


实验结果表明,三种算法除准时覆盖率$R_o$和有效覆盖率$R_e$的结果有明显差异外,改进后的遗传算法总飞行时间和总飞行里程下降了约3.6\%,且两种遗传算法的总飞行时间和总飞行里程均低于带载重的贪婪成本选择算法,
这符合课题中优先提高准时覆盖率$R_o$和有效覆盖率$R_e$的同时也要考虑降低无人机总飞行时间和总飞行里程,以降低无人机救援成本的目标。

实验组(三):待救援区域为边长50km的正方形,基地$B$位于(0,0),无人机的数量为8架,兴趣点的数量为100,无人机的飞行速度为25m/s,最大载重为100kg,兴趣点的时间敏感性随机范围为(80mins,150mins),物资需求随机范围为(5kg,15kg)。遗传算法的迭代次数从0增加到50。
\begin{itemize}
    \item [(1)]可以看到,如\figref{fg503}所示,在迭代次数小于20次时,此时改进前后遗传算法的准时覆盖率$R_o$和有效覆盖率$R_e$没有明显的差异,
    而当迭代次数逐次增加,改进的遗传算法拥有更好的覆盖率,且改进后的遗传算法在40次迭代后已经得到最优解,而改进前的遗传算法需要进行50次迭代后才能得到最优解。
    \begin{figure}[H]
    \begin{center}
        \begin{tikzpicture}
            \begin{axis}
                %坐标系配置,只显示左边坐标轴
                [axis lines = left, xlabel = {迭代次数}, ylabel = {覆盖率},tick align=outside,legend style={at={(1.5,0.75)},anchor=north}]
                %绘制曲线-1
                \addplot+[smooth,red,mark=triangle]
                coordinates{
                    (0,0.1425) (10,0.2242) (20,0.5245)
                    (30,0.6741) (40,0.7545)(50,0.7912)
                };
                \addlegendentry{经典遗传算法$\enspace R_o$}; %绘制标注
                %绘制曲线-2
                \addplot+[smooth,orange,mark=triangle]
                coordinates{
                    (0,0.1523) (10,0.2612) (20,0.5924)
                    (30,0.7541) (40,0.8645)(50,0.8702)
                };
                \addlegendentry{带自交的遗传算法$\enspace R_o$}; %绘制标注
                %绘制曲线-3
                \addplot+[smooth,blue,mark=*]
                coordinates{
                    (0,0.1425) (10,0.3442) (20,0.6545)
                    (30,0.8041) (40,0.9345)(50,1)
                };
                \addlegendentry{经典遗传算法$\enspace R_e$}; %绘制标注
                %绘制曲线-4
                \addplot+[smooth,purple,mark=*]
                coordinates{
                    (0,0.1537) (10,0.3612) (20,0.6924)
                    (30,0.8541) (40,1)(50,1)
                };
                \addlegendentry{带自交的遗传算法$\enspace R_e$}; %绘制标注
            \end{axis}
        \end{tikzpicture}
    \end{center}
    \caption{三种算法的覆盖率随迭代次数的变化情况}
    \label{fg503}
\end{figure}
    \item [(2)]在该组实验参数下,由于有8架无人机参与救援物资运输,故一共会产生8条无人机的配送路径,这里需要统计在迭代过程中这8条路径违反载重约束的情况。实验结果如\figref{fg504}所示。
\begin{figure}[H]
    \begin{center}
        \begin{tikzpicture}
            \begin{axis}
                %坐标系配置,只显示左边坐标轴
                [axis lines = left, xlabel = {迭代次数}, ylabel = {违反载重约束线路数量},tick align=outside,legend style={at={(1.5,0.75)},anchor=north}]
                %绘制曲线-1
                \addplot+[smooth,red,mark=triangle]
                coordinates{
                    (0,7) (5,5) (10,6)
                    (15,4) (20,3)(25,4)
                    (30,2) (35,1)(40,1)
                    (45,0) (50,0)
                };
                \addlegendentry{经典遗传算法$\enspace$违约线路数量}; %绘制标注
                %绘制曲线-2
                \addplot+[smooth,orange,mark=triangle]
                coordinates{
                    (0,7) (5,5) (10,4)
                    (15,1) (20,2)(25,1)
                    (30,0) (35,0)(40,0)
                    (45,0) (50,0)
                };
                \addlegendentry{带自交的遗传算法$\enspace$违约线路数量}; %绘制标注
            \end{axis}
        \end{tikzpicture}
    \end{center}
    \caption{改进前后遗传算法违约路径数量随迭代次数的变化情况}
    \label{fg504}
\end{figure}
\qquad 从实验结果可以得出结论,由于改进的遗传算法在进化过程中加入了符合“优胜劣汰”原则的多轮筛选过程,
使得违反载重约束的个体不易进入下一代参与进化,故改进后的遗传算法能在较少的迭代次数内将违反载重约束的降至0。
\end{itemize}
\section{本章小结}[Brief summary]
本章主要对带时间敏感性的无人机网络扫描覆盖问题进行了实验测试。根据对现实生活中实际情况的调查和研究,
设置了实验参数,并用带载重的贪婪成本选择算法、经典遗传算法和带自交的遗传算法进行了求解。
通过对实验结果的分析,本文发现两种遗传算法的有效覆盖率$R_e$均比带载重的贪婪成本选择算法高,且无人机飞行成本(以飞行里程和时间计)均比带载重的贪婪成本选择算法低。
同时,改进后的带自交的遗传算法的有效覆盖率$R_e$比改进前高,无人机飞行成本比改进前低。

