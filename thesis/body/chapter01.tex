% !TEX root = ../main.tex

% 中英标题:\chapter{中文标题}[英文标题]
\chapter{绪论}[Introduction]

\section{课题背景及研究的目的和意义}[Background, objective and significance of the subject]

% 正文内容,注意LaTeX分段有两种方法,直接空一行或者使用<\par>
% 默认首行缩进,不需要在代码编辑区手动敲空格
我国的自然灾害发生较为频繁,带来了严重的社会危害。据统计,中国是世界上自然灾害发生频率最高,且造成后果最严重的的少数几个国家之一\cite{ziranzaihaifangzhide}。中国的自然灾害种类多,发生频率较高,且灾情十分严重。
在自然灾害产生后,首要任务是进行救援和救灾。在重大自然灾害发生后,不仅会给社会经济造成较大的损失,更严重的情况下可能会造成受灾地区通信中断和物资供应中断,甚至导致威胁人身安全。研究表明,
自然灾害发生后的黄金救援期为72小时,而灾区需要的物资多为紧急需要的医疗物资和生活物资等,因此顺利且准时地将救援物资投放到受灾地区成为了衡量救灾任务是否成功的决定性因素之一\cite{yuqingqing}。
高效且准确的应急救援行动可以提高物资的运输效率,同时减少救援所需要的时间,从而有效降低自然灾害造成的损失,保护人民群众的生命财产安全。


自然灾害的发生属于突发性事件,是难以避免的,但是通过对应急救援场景的分析,从而确定符合实际需求的应急救援物资调度方案,可以确保救援工作的高效进行和救援物资的及时合理分配。
无人机(Unmanned Aerial Vehicle,UAV)作为无人驾驶飞行器,具有实时性强、不容易受外界环境影响、部署条件较为灵活和使用维护成本较为低廉等特点。
突发事件的种类具有多样性,灾区现场的复杂环境导致与采用人工救援相比,使用多架无人机组成的无人机网络协同进行救援工作有着较大的优势。采用无人机网络协同作业代替人工作业,能够有效减少救援过程中的意外事故风险,同时能够节省人力资源。
目前,我国的无人机工业发展迅速,无人机在救援救灾领域中已得到较为广泛的应用。在货运车辆受限于环境无法抵达灾区时,无人机能部分替代货运车辆,抢先把急需的重要物资运输到受灾地区\cite{liuyajing}。


在确定了使用无人机网络进行多无人机协同救援工作后,需要对无人机的飞行路径进行合理的规划,以满足救援工作的实际需求。在无人机数量和待救援地点数量的影响之下,救援任务的分配也变得较为复杂。
随着无人机网络协同工作技术的发展,救灾的前期准备工作时间得以大幅较少,从而使得救灾任务的效率得到提升。


本文以灾区环境下针对突发自然灾害的多无人机组成的无人机网络协同调度救灾问题为背景开展研究,建立带时间敏感性的无人机网络扫描覆盖问题的模型,研究目标是在满足各灾区救援点时间敏感和无人机的载重上限条件下,
如何尽可能地提高无人机网络中无人机对待救援地点的扫描覆盖率,同时使得救援成本尽可能维持在较低水平。


\section{国内外研究现状}[Current research state]

\subsection{无人机网络在应急救灾方面的应用}[Application of UAV in emergency relief]
2008年,国产千里眼无人机对四川汶川大地震后的北川县城进行了拍摄,成为了抗震救灾过程中的重要参考资料\cite{2010dizhen}。此后,近年来随着新兴科技成果的迅速转化和对灾害救援要求的不断提高,无人机网络在防灾、减灾和救灾领域的应用越来越频繁和多样。
凭借着无人机的高灵活性和高机动型,其在应急救灾领域始终发挥着重要的作用。目前,国内外已有不少企业将目光投向无人机技术。在2013年,Amazon就开始尝试使用无人机进行物流配送,并且Amazon推出了新型混合动力物流无人机Prime Air Drone,该款无人机支持垂直起降,
续航里程可达30公里,除此之外Amazon还为其开发了计算机视觉识别技术,可以使无人机自动寻找合适的降落点\cite{zhanghonghai}。而在国内企业中,顺丰速运和京东作为物流行业中的领军企业,均较早提出了无人机网络用于物流和救灾方面的构想。顺丰主张主线路高容量运载自动驾驶飞机、分支线路中运载量自动驾驶飞机和末端物资配送自动驾驶飞机结合的三段式空运网络。目前,无人机应急救灾和配送的研究已经十分成熟。将无人机应用到应急救灾领域中,可以充分配合其他工具,从而实现“1+1>2”的效果。在2020年初新型冠状病毒感染的肺炎疫情期间,顺丰速运在武汉市、温州市和哈尔滨市等地投入了大量无人机,主要负责运送
防护服、口罩、手套、生活物资、药品等继续的救援物资,其在上述城市的单日运送总量达到1.8吨,飞行总里程超过1500公里。顺丰速运表示,顺丰的无人机主要用途是将货物运送至较难进行配送覆盖的偏远地区,这与将无人机网络应用于应急救灾领域的构想不谋而合。2020年,京东也推出了自主研发的物流无人机,载重可高达数百公斤,可用于应急救援时的大规模物资运输投放,提高救援效率\cite{renxuan}。


表\ref{table1}是部分物流无人机的参数信息。
  \begin{table}[htbp]
    \vspace{0.5em}\centering\wuhao
    \caption{部分物流无人机参数}\label{table1}
    \begin{tabular}{cccc}
    \toprule[1.5pt]
    公司 & 型号 & 最大载重 & 最大续航里程 \\
    \midrule[1.5pt]
    亚马逊 & Prime Air Drone & 2.5kg & 30km \\
    顺丰 & Ark & 12kg & 20km \\
    顺丰 & Manta Ray & 10kg & 100km \\
    京东 & 京蜓 & 数百千克 & 未公布 \\ 
    \bottomrule[1.5pt]
    \end{tabular}
    \end{table}

\subsection{无人机网络中路径规划问题研究现状}[Research state of UAV routing problems]
扫描覆盖作为无线传感器网络(Wireless Sensor Networks, WSN)中的一个新兴领域,近年来受到了广泛的关注。通常情况下,根据问题的最终优化目标,扫描覆盖问题可分为以下三个类型:
\begin{itemize}
  \item [(1)] 传感器数量的优化。这类问题的目标是最小化区域内扫描覆盖范围的移动传感器数量。解决这类问题的一般出发点是较少移动传感器在目标区域内的移动距离。
  Cheng等人在文献中证明,解决该问题的算法是NP-Hard的,并据此提出了一种称为CSWEEP的启发式算法\cite{2008Sweep};
  \item [(2)] 扫描覆盖时间周期的优化。该问题的目标是最小化对区域内所有兴趣点进行扫描覆盖所投入的时间。Feng等人在文献中提出了目标是最小化扫描覆盖总时间(M$^3$SR)的问题\cite{2015Shorten};
  \item [(3)] 传感器传输延迟的优化。该问题的目标是降低数据中心与移动传感器通信过程中的通信延迟,Zhao等人在文献中提出了有啊传感器延迟的解决方案\cite{2012zhao}。
\end{itemize}


与上述研究领域的目标不同,本文主要研究的是无人机网络中的路径规划问题。其与车辆路径规划问题(Vehicle Routing Problem, VRP)较为相似,因此在研究时可以参考车辆路径问题领域的相关方法。
在历来的研究中,使用同一型号车辆的车辆调度问题认为是旅行商问题(Traveling Salesman Problem, TSP)的特殊情况\cite{bektas2006multiple}。同时研究表明,该问题是一个NP-hard问题\cite{lenstra1981complexity}。


随着车辆路径规划问题在实际应用过程中的不断发展,其又产生了许多不同目标和不同领域的扩展,包括多型号车辆的车辆路径规划问题\cite{pan2021multi}、随机需求的车辆路径规划问题\cite{saint2021time}、单一车辆多次投入使用的车辆路径规划问题\cite{euchi2021hybrid}和带收集的车辆路径规划问题\cite{fan2021time}等。
而本文中所涉及的车辆路径规划问题被称为带时间窗的车辆路径规划问题(Vehicle Routing Problems With Time Windows, VRPTW)。在带时间窗的车辆路径规划问题中,兴趣点拥有时间窗限制,车辆需要在时间窗内访问兴趣点,并找到对兴趣点实现最大准时覆盖率的路径\cite{ren2022vehicle}。


对于NP-hard问题,使用精确算法进行求解并不现实,因此目前解决VRPTW问题的主要方法为启发式方法。较为简单的启发式方法其的代表类型是贪婪式算法,在使用贪婪式算法求解问题时一般选择当前最优解,而非考虑整体情况的最优解,其较为简单,且拥有求解速度快的优点,但是只适合求解问题规模较小的NP-hard问题。
而智能优化算法主要包括模拟退火算法(Simulated Annealing, SA)、蚁群算法(Ant Colony Optimization, ACO)、禁忌搜索算法(Tabu Search, TS)和遗传算法(Genetic Algorithm,GA)等。模拟退火算法能够快速收敛,是一种全局搜索算法\cite{lee2021simulated}。蚁群算法根据蚂蚁的觅食行为进行设计,蚂蚁的觅食路径就是优化问题的解决方案。
蚁群内的蚂蚁可以通过某种信息机制实现信息的传递,其会在其经过的路径上释放一种可以称之为“信息素”的物质,蚁群会沿着“信息素”浓度较高路径行走,在正反馈机制下,蚁群会集中在信息素浓度较大的路径上,从而得到了问题的最优解\cite{wu2021hybrid}。禁忌搜索算法最先被Willard等人应用到车辆路径规划问题中\cite{ilhan2021improved},属于局部搜索算法。遗传算法根据大自然中生物的进化规律设计,模拟了达尔文生物进化论中的自然选择和遗传学中的生物进化过程,在求解较复杂的组合优化问题时,能较快地获得较好的优化结果。


而带时间窗的无人机网络路径规划问题(Unmanned Aerial Vehicle Routing Problems With Time Windows, UAVRPTW)作为带时间窗的车辆路径规划问题的变种,目前也在持续发展中。Choi等人在文献中提出了一种针对无人机续航有限问题的新型路径规划问题,他们将无人机的扫描覆盖任务定义为起飞、巡航、悬停、转弯和着陆五个阶段,并且提出了新的路径优化模型\cite{2019Energy}。Li等人在文献中提出了无人机的最短时间最大覆盖问题(MTMC),即在无人机性能的约束下,
在给定的区域内,实现在最短的时间内对兴趣点实现最大覆盖的目标,他们考虑了无人机的性能,并进行了数学建模,将多目标优化问题转化为单目标路径规划问题,取得了较好的优化结果\cite{2020A}。Thibbotuwawa等人在文献中提出了带无人机充电中心的UAVRPTW数学模型和算法,解决了无人机飞行的局限性\cite{thibbotuwawa2020unmanned}。


然而,过往的研究较多忽视了无人机返回基地前需要保留一定续航时间的问题,且较少考虑无人机与传统车辆存在的物理特性差异。本文对这些问题作出了改进,确保了无人机能够在续航时间结束前返回基地,并考虑了无人机的载重问题。
\section{本文的主要研究内容}[Main research contents of this subject]

本课题的主要研究目标是解决带时间窗的无人机网络扫描覆盖路径规划问题,并根据问题的解决算法设计出一套无人机网络紧急救援模拟系统。

本文的主要工作包括:
\begin{itemize}
  \item [(1)] 
  带时间敏感性的无人机网络扫描覆盖问题的建立。


  \qquad 首先针对应急救援过程中的关键要素,如无人机基地、兴趣点、无人机网络和无人机等进行分析,然后需要分析使用无人机网络进行多架无人机协同应急救援的全部流程,找到提高无人机飞行效率的方法和解决问题的整体流程。
  针对无人机网络应急救援问题,结合无人机自身的物理特性约束,建立了单目标的无人机应急救援问题模型,并针对该模型进行了参数的假设。在此基础上,设计了模型的数学表达式和问题的优化目标。
  \item [(2)]
  带时间敏感性的无人机网络扫描覆盖问题的算法设计。


  \qquad 根据问题模型,设计了问题解的表达方式和初始解的产生方法。本文主要采用两种启发式算法——带载重的贪婪成本选择算法和带自交的遗传算法进行实验和求解。
  \item [(3)]
  算法性能的评估与对比。


  \qquad 根据对实际场景的分析,设置了问题模拟中的场景参数和算法参数,对模型进行求解。得到结果后,还需要对实验结果进行分析,主要需要分析几种算法的可行性和性能优劣,验证通过改进后的算法能否得到求解问题模型的预期结果。
  \item [(4)]
  无人机网络紧急救援系统的模拟实现。


  \qquad 在得到问题求解算法后,需要基于该算法为原型,进行相应的技术选型,并分析任务目标和任务需求,设计系统的总体框架和各模块功能,并对系统进行模拟实现。
\end{itemize}