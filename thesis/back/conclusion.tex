% !TEX root = ../main.tex

% 结论
\begin{conclusions}
自然环境的不断变化和无人机等高新技术的发展使得应急救援领域有了新的挑战和新的工具,面对传统运输车辆无法顺利进行任务的复杂环境,
无人机成为了一个重要的紧急物资运输工具的选择。而在使用无人机进行紧急救援物资投放时,如何规划高覆盖率的无人机路径成为了一个重要问题。
本文通过对带时间窗的车辆路径规划问题相关文献的研究,在传统VRPTW问题的基础上,考虑了无人机作为运输载具的特殊性,
以提升无人机网络中扫描覆盖问题的有效覆盖率为目标,对带时间敏感性的无人机网络扫描覆盖问题进行了研究。


本文的主要工作如下:
\begin{itemize}
	\item [(1)] 针对无人机参与的应急救援情况下物资运输问题,建立了无人机紧急救援系统的问题模型。在该模型中,将待救援地点统称为兴趣点,所有无人机从基地出发,按照规划路径对兴趣点进行覆盖,最后返回无人机基地,完成任务。
 	\item [(2)] 提出了带载重的贪婪成本选择算法和带自交的遗传算法来求解问题模型。为了提高算法性能,在原算法基础上进行了相应的改进,以帮助寻求问题的最优解。同时进行了仿真实验,评估了算法的性能。实验结果表明,带载重的贪婪成本选择算法和带自交的遗传算法均能有效解决这一问题模型,且带自交的遗传算法的有效覆盖率$R_e$更高,无人机的飞行成本更低。
\end{itemize}


本文在一定程度上实现了开题时的设定目标,但在后续的研究中,发现本文的研究成果还存在以下问题:
\begin{itemize}
	\item [(1)] 本文在研究救援物资时忽略了其体积。在现实场景中,不同物资的规格相差较大,可能会影响无人机的运输效率,该部分还有待未来的进一步研究。
 	\item [(2)] 本文在研究无人机飞行过程时,认为无人机会以恒定速度和恒定续航里程飞行。但是在现实场景中,无人机的飞行速度和续航里程会受自然环境、载重和航向等多重因素的共同影响,在下一步研究中希望能分析无人机飞行速度和续航时间与上述因素之间的关系,而非完全理想化使用固定的速度和续航里程。
\end{itemize}

\end{conclusions}